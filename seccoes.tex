\divider
\section{Resumo}
  \vspace{-0.2cm}
  
  %Bom senso de responsabilidade, assiduidade e pontualidade, adaptação e flexibilidade, por fim e não menos importante um elevado espírito de trabalho em equipa. Poderá consultar informações mais detalhadas sobre as minhas experiências profissionais, formação académica e formações complementares nas secções a seguir.
  
  
  Bom senso de responsabilidade, assiduidade e pontualidade, adaptação e flexibilidade, por fim e não menos importante um elevado espírito de trabalho em equipa. Poderá consultar informações mais detalhadas sobre as minhas experiências profissionais, formação académica e formações complementares nas secções a seguir.\\
  \divider

%----------------------------------------------------------------------------------------
%	Secção Experiências Profissional
%----------------------------------------------------------------------------------------
\section{Experiências}

\begin{entrada}

\lista
  {Actualmente}
  {Instituto Superior Politécnico do Huambo - (UJES)}
  {Huambo, Angola}
  {\textsf{\emph{Professor Assistente Estagiário}}\\
  \small{Leciono as disciplinas de Programação Orientada a Objectos}, Programação I e Linguagens de Programação.}
%------------------------------------------------

\lista
  {2018 - 2019}
  {Instituto Superior de Humanidades Tecnológicas EKUIKUI II - (ISUPE)}
  {Huambo, Angola}
  {\textsf{\emph{Professor Assistente Estagiário}}\\
  \small{Lecionei as disciplinas de Programação Web e Programação Orientada a Objectos.}}
%------------------------------------------------
\lista
  {2016}
  {Governo Provincial do Huambo}
  {Huambo, Angola}
  {\textsf{\emph{Técnico de Informática}}\\
  \small{Trabalhei como Técnico de Informática no Gabinete do Vice – Governador P/ Serviços Técnicos e Infraestruturas}}
%------------------------------------------------

%------------------------------------------------


\end{entrada}

%----------------------------------------------------------------------------------------
%	Secção Educação
%----------------------------------------------------------------------------------------
\section{Educação}

\begin{entrada}

\lista
{2016--2018}
{Mestrado {\normalfont em Engenharia Informática }}
{Universidade da Beira Interior - (UBI), Portugal}
{\vspace{-0.3cm}}

\dissert
{}
{\textbf{Tema:} \mSiteSquare{http://hdl.handle.net/10400.6/10019}{Mapeamento de Requisitos de Segurança à Tecnologias na Internet das Coisas}}
{Orientador: {\normalfont \mSiteSquare{www.di.ubi.pt/~inacio/}{Prof. Doutor. Pedro Ricardo Morais Inácio}}}
{Oponente: {\normalfont \mSiteSquare{www.dei.estg.ipleiria.pt/mgsim/pt-PT/equipa/corpo-docente/carlos-rabadao.php}{Prof. Doutor. Carlos Manuel da Silva Rabadão}}}
{\vspace{-0.3cm}}



\lista
{2010--2014}
{Licenciatura {\normalfont em Engenharia Informática }}
{Instituto Superior Politécnico do Huambo - (UJES)}
{\vspace{-0.3cm}}

\dissert
{}
{\textbf{Tema:} Desenvolvimento de um sistema de Gestão de Estoque e Faturação para a Empresa O\&M PC - Prestação de Serviços, Lda}
{Orientador: {\normalfont Prof. Eng. Félix Vladimir Roldán Jiménez}}
{Oponente: {\normalfont Prof. Eng. Yordanis Arencibia}}
{\vspace{-0.3cm}}

%------------------------------------------------
\end{entrada}
%----------------------------------------------------------------------------------------
% Secção Projetos
%----------------------------------------------------------------------------------------
\vspace{-1.3cm}
\newpage
\section{Projectos Open Source}
\begin{entrada}
  \project
  { \emph{[ App-Mobile ]} }
  {Covid19angola}
  {\textsf{\emph{[\mSiteSquare{github.com/moser-jose/covid19angola}{https://github.com/moser-jose/covid19angola}]}}\\
  \small{App Mobile para as estatísticas do Covid-19 em Angola e ao redor do mundo.}}

  \project
  { \emph{[ App-Mobile ]} }
  {Hinário Adventista do 7º Dia}
  {\textsf{\emph{[\mSiteSquare{hinario-adventista.vercel.app}{https://hinario-adventista.vercel.app}]}}\\
  \small{App Mobile que trás todos os hinos do hinário da igreja Adventista do Sétimo Dia.}}

  \project
  { \emph{[ Theme ]} }
  {mosmmy-theme-vscode}
  {\textsf{\emph{[\mSiteSquare{moser-jose.github.io/mosmmy-theme-vscode/}{https:moser-jose.github.io/mosmmy-theme-vscode/}]}}\\
  \small{Tema para o vscode (cores e fonte de letras).}}

  \project
  { \emph{[ Theme ]} }
  {mosmmy-icons-vscode}
  {\textsf{\emph{[\mSiteSquare{github.com/moser-jose/mosmmy-icons-vscode}{https:github.com/moser-jose/mosmmy-icons-vscode}]}}\\
  \small{Ícones para os ficheiros e pastas para o vscode.}}

\end{entrada}
%----------------------------------------------------------------------------------------
%	Secção Cursos Profissionais
%----------------------------------------------------------------------------------------
%\vspace{-1.3cm}
%\newpage
\section{Cursos Profissionais}

\begin{entrada}

%------------------------------------------------
%\cursos
%{2020}
%{React Completo}
%{\mSiteSquare{hdl.handle.net/10400.6/10019}{ORIGAMID}}
%{\vspace{-0.3cm}}
%------------------------------------------------

%------------------------------------------------
%\cursos
%{2020}
%{Vue.js Completo}
%{\mSiteSquare{hdl.handle.net/10400.6/10019}{ORIGAMID}}
%{\vspace{-0.3cm}}
%------------------------------------------------


%------------------------------------------------
\cursos
{2019}
{Construção e Lançamento de Pequenos Satélites (CANSAT)}
{Gab. de Gest. do Prog. Esp. Nacional (GGPEN)}
{\vspace{-0.3cm}}
%------------------------------------------------

%------------------------------------------------
\cursos
{2017}
{Linux Essentials}
{Universidade da Beira Interior, Portugal}
{\vspace{-0.3cm}}
%------------------------------------------------
\cursos
{2016}
{Seminário de Capacitação Pedagógica}
{Instituto Superior Politécnico do Huambo}
{\vspace{-0.3cm}}

%------------------------------------------------
\cursos
{2009}
{Reparação e Montagem de Computadores}
{MAPESS, Luanda}
{\vspace{-0.3cm}}
%------------------------------------------------
\end{entrada}

%----------------------------------------------------------------------------------------
%	Secção Publicação
%----------------------------------------------------------------------------------------
\section{Publicações}

%\begin{entrada}
  %------------------------------------------------
  
 % \public
  %{2020}
  %{Artigo de Revista}
  %{Autor}
  %{\textbf{Titulo:} Um estudo comparativo sobre a performance de mecanismos de segurança em dispositivos da Internet das Coisas.}
  %{\textbf{Resumo:} O artigo faz um estudo sobre a performance de mecanismos de segurança em plataformas de desenvolvimento atuais da IdC, especificamente num Raspberry Pi 3, assumindo que este conhecimento é importante para o desenho de sistemas e softwares seguros para a IdC.}
  %{\textbf{Revista:} RAC - Revista Angolana de Ciências | ISSN:2664-259X}
  %{}{https://doi.org/10.1145/3339252.3340514}
  %{\vspace{-0.3cm}}
  %------------------------------------------------
  %\end{entrada}

\begin{entrada}
%------------------------------------------------

\public
{2019}
{Conference Proceesings}
{Co-Author}
{\textbf{Title:} IoT-HarPSecA: A Framework for facilitating the Design and Development of Secure IoT Devices.}
{\textbf{Abstract:} The security framework is aimed at facilitating the choice of specific security algorithms given a set of security goals, hardware specifications, message payload size, application area, and energy requirement.}
{\textbf{Conference:} Proceedings of the 14th International Conference on Availability, Reliability and Security (ARES 2019)}
{}{https://doi.org/10.1145/3339252.3340514}
{\vspace{-0.3cm}}
%------------------------------------------------
\end{entrada}

%----------------------------------------------------------------------------------------
%	Secção Interesses
%----------------------------------------------------------------------------------------
\section{Interesses}
  \mTag{Programming languages}
  \mTag{Computer network}
  \mTag{Internet of Thinks - (IoT)}\\
  \mTag{Deep Learning}
  \mTag{Cryptoanalysis}
  \mTag{Information security}
  \mTag{Big Data and Data mining}\\
  \mTag{Cloud Computing}
  \mTag{Systems development}\\
  \divider

%----------------------------------------------------------------------------------------
%	Secção Referências
%----------------------------------------------------------------------------------------

\section{Referências}
 
\begin{reflist}

  \referencia
  
  {Prof. Felisberto Fato, MSc}
  {Instituto Superior Politécnico do Huambo, Angola}
  {\jobtitle{Chefe do Departamento das TIC'S}\\
  \mEmail{f.fc.fato@gmail.com}{f.fc.fato@gmail.com}}
  
  \referencia
 
  {Prof. Ilson Gaspar Ferreira, MSc}
  {Instituto Superior Politécnico do Huambo, Angola}
  {\jobtitle{Coordenador do Curso de Engenharia Informática e Computadores}\\
  \mEmail{ilsongaspar10@hotmail.com}{ilsongaspar10@hotmail.com}}

  
\end{reflist}
%----------------------------------------------------------------------------------------