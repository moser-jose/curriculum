\divider
\section{Resumo}
  \vspace{-0.2cm}
  
  Bom senso de responsabilidade, assiduidade e pontualidade, adaptação e flexibilidade, por fim e não menos importante um elevado espírito de trabalho em equipa. Poderá consultar informações mais detalhadas sobre as minhas experiências profissionais, formação académica e formações complementares nas secções a seguir.\\
  \divider

%----------------------------------------------------------------------------------------
%	Secção Experiências Profissional
%----------------------------------------------------------------------------------------
\section{Experiências}

\begin{entrada}

\lista
  {Actualmente}
  {Instituto Superior Politécnico do Huambo}
  {Huambo, Angola}
  {\textsf{\emph{Professor Assistente Estagiário}}\\
  \small{Leciono as disciplinas de Programação Orientada a Objectos}, Programação I e Linguagens de Programação.}
%------------------------------------------------

\lista
  {2016 - 2018}
  {Instituto Superior de Humanidades Tecnologicas EKUIKUI II - (ISUPE)}
  {Huambo, Angola}
  {\textsf{\emph{Professor Assistente Estagiário}}\\
  \small{Lecionei as disciplinas de Programação Web e Programação Orientada a Objectos.}}
%------------------------------------------------
\lista
  {2016}
  {Governo Provincial do Huambo}
  {Huambo, Angola}
  {\textsf{\emph{Técnico de Informática}}\\
  \small{Trabalhei como Técnico de Informática no Gabinete do Vice – Governador P/ Serviços Técnicos e Infraestruturas}}
%------------------------------------------------

%------------------------------------------------


\end{entrada}

%----------------------------------------------------------------------------------------
%	Secção Educação
%----------------------------------------------------------------------------------------
\section{Educação}

\begin{entrada}
%------------------------------------------------
%\entry
%{2016--2018}
%{Mestrado {\normalfont em Engenharia Informática}}{Universidade da Beira %Interior, Portugal}
%{\vspace{-0.3cm}}

%\entryL
%{}
%{\textbf{Tema:} Mapping Security Requeriments with Technology in the Internet of Things}
%{Orientador: {\normalfont Prof. Doutor Pedro Ricardo Morais Inácio}}
%{Oponente: {\normalfont Prof. Doutor Andre Pedro João}}
%{\vspace{-0.3cm}}
%------------------------------------------------

\lista
{2016--2018}
{Mestrado {\normalfont em Engenharia Informática }}
{Universidade da Beira Interior, Portugal}
{\vspace{-0.3cm}}

\dissert
{}
{\textbf{Tema:} Mapeamento de Requisitos de Segurança à Tecnologias na Internet das Coisas}
{Orientador: {\normalfont \mSiteSquare{www.di.ubi.pt/~inacio/}{Prof. Doutor. Pedro Ricardo Morais Inácio}}}
{Oponente: {\normalfont \mSiteSquare{www.dei.estg.ipleiria.pt/mgsim/pt-PT/equipa/corpo-docente/carlos-rabadao.php}{Prof. Doutor. Carlos Manuel da Silva Rabadão}}}
{\vspace{-0.3cm}}



\lista
{2010--2014}
{Licenciatura {\normalfont em Engenharia Informática }}
{Instituto Superior Politécnico do Huambo}
{\vspace{-0.3cm}}

\dissert
{}
{\textbf{Tema:} Desenvolvimento de um sistema de Gestão de Estoque e Faturação para a Empresa O\&M PC - Prestação de Serviços, Lda}
{Orientador: {\normalfont Prof. Eng. Félix Vladimir Roldán Jiménez}}
{Oponente: {\normalfont Prof. Eng. Yordanis Arencibia}}
{\vspace{-0.3cm}}

%------------------------------------------------
%\entry
%{2006--2009}
%{Técnico Médio {\normalfont em Informática}}
%{Complexo Escolar Politécnico Elsamina, Luanda}
%{\vspace{-0.3cm}}
%------------------------------------------------
\end{entrada}

%----------------------------------------------------------------------------------------
%	Secção Cursos Profissionais
%----------------------------------------------------------------------------------------
\vspace{-1.3cm}
\newpage
\section{Cursos Profissionais}

\begin{entrada}
%------------------------------------------------
%\cursos
%{2020}
%{Vue.js Completo}
%{\mSiteSquare{www.origamid.com/curso/vue-js-completo/}{ORIGAMID}}
%{Centro de Formação Tecnológico do ITEL}
%{\vspace{-0.3cm}}
%------------------------------------------------
%------------------------------------------------
%\cursos
%{2020}
%{Web Design Completo}
%{\mSiteSquare{www.origamid.com/curso/web-design-completo}{ORIGAMID}}
%{Centro de Formação Tecnológico do ITEL}
%{\vspace{-0.3cm}}
%------------------------------------------------

%------------------------------------------------
%\cursos
%{2020}
%{JavaScript Completo ES6}
%{\mSiteSquare{www.origamid.com/curso/javascript-completo-es6}{ORIGAMID}}
%{Centro de Formação Tecnológico do ITEL}
%{\vspace{-0.3cm}}
%------------------------------------------------


%------------------------------------------------
%\cursos
%{2020}
%{Bootstrap 4}
%{\mSiteSquare{www.origamid.com/curso/bootstrap-4}{ORIGAMID}}
%{Centro de Formação Tecnológico do ITEL}
%{\vspace{-0.3cm}}
%------------------------------------------------

%------------------------------------------------
\cursos
{2019}
{Construção e Lançamento de Pequenos Satélites (CANSAT)}
{Gab. de Gest. do Prog. Esp. Nacional (GGPEN)}
%{Centro de Formação Tecnológico do ITEL}
{\vspace{-0.3cm}}
%------------------------------------------------

%------------------------------------------------
\cursos
{2017}
{Linux Essentials}
{Universidade da Beira Interior, Portugal}
{\vspace{-0.3cm}}
%------------------------------------------------
\cursos
{2016}
{Seminário de Capacitação Pedagógica}
{Instituto Superior Politécnico do Huambo}
{\vspace{-0.3cm}}

%------------------------------------------------
\cursos
{2009}
{Reparação e Montagem de Computadores}
{MAPESS, Luanda}
{\vspace{-0.3cm}}
%------------------------------------------------
\end{entrada}

%----------------------------------------------------------------------------------------
%	Secção Publicação
%----------------------------------------------------------------------------------------
\section{Publicações}

\begin{entrada}
%------------------------------------------------

%\public
%{2021}
%{Livro}
%{Autor}
%{\textbf{Título:} Seja diferente mas não indiferente}
%{\textbf{Resumo:} O Livro apresenta um vislumbre sobre o preconceito que temos das pessoas em relação o que são. De igual modo, incentiva as mesmas à analisarem as coisas tendo em consideração a perspectiva dos outros.}
%{\textbf{Editora:} Mosmmy Editora, Vol. I}
%{Livro}{https://v2.overleaf.com/project/5b01c0437541e761f78e6927}
%{\vspace{-0.3cm}}

\public
{2019}
{Artigo de Conferência}
{Co-Author}
{\textbf{Title:} IoT-HarPSecA: A Framework for facilitating the Design and Development of Secure IoT Devices.}
{\textbf{Abstract:} The security framework is aimed at facilitating the choice of specific security algorithms given a set of security goals, hardware specifications, message payload size, application area, and energy requirement.}
{\textbf{Conference:} Proceedings of the 14th International Conference on Availability, Reliability and Security (ARES 2019)}
{}{https://doi.org/10.1145/3339252.3340514}
{\vspace{-0.3cm}}
%------------------------------------------------
\end{entrada}


%Within the scope of this framework, we develop an easy-to-use tool in C++ that allows users to interact with the IoT-HarPSecA framework.

%----------------------------------------------------------------------------------------
%	AWARDS SECTION
%----------------------------------------------------------------------------------------
%\section{Prémios}

%\begin{entrylist}
%------------------------------------------------
%\entry
%{2014}
%{Award name}
%{Institution}
%{Award description. Award description. Award description. Award description. Award description. Award description. Award description. }
%------------------------------------------------
%\end{entrylist}

%----------------------------------------------------------------------------------------
%	Secção Interesses
%----------------------------------------------------------------------------------------
\section{Interesses}
  \mTag{Linguagens de Programação}
  \mTag{Redes de Computadores}
  \mTag{Internet das Coisas - (IoT)}\\
  \mTag{Deep Learning}
  \mTag{Criptoanálise}
  \mTag{Segurança da Informação}
  \mTag{Big Data e Data mining}\\
  \mTag{Cloud Computing}
  \mTag{Desenvolvimento de Sistemas}\\
  \divider

%----------------------------------------------------------------------------------------
%	Secção Referências
%----------------------------------------------------------------------------------------

\section{Referências}
 
\begin{reflist}

  \referencia
  
  {Prof. Felisberto Fato}
  {Instituto Superior Politécnico do Huambo, Angola}
  {\jobtitle{Chefe do Departamento das TIC'S}\\
  \mEmail{f.fc.fato@gmail.com}{f.fc.fato@gmail.com}}
  
  \referencia
 
  {Prof. Lissete Monteiro Herrera}
  {Instituto Superior Politécnico do Huambo, Angola}
  {\jobtitle{Coordenadora do Curso de Engenharia Informática}\\
  \mEmail{lmonteroherrera@hotmail.com}{lmonteroherrera@hotmail.com}}

  
\end{reflist}

%---------
%----------------------------------------------------------------------------------------